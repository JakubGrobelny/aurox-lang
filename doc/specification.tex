\documentclass[12pt]{article}
\usepackage[polish]{babel}
\usepackage[T1]{fontenc}
\usepackage[utf8]{inputenc}
\usepackage{syntax}

\setlength{\grammarindent}{5em}

\title{\huge
    \textbf{Statycznie typowany funkcyjny język programowania o ewaluacji leniwej}}
\date {Kurs języka Prolog (Q1) -- projekt końcowy}
\author{\Large Jakub Grobelny}

\begin{document}

\maketitle

\section{Struktura leksykalna i składnia}

Klasyfikacja znaków:
\begin{itemize}
    \item \textit{biały znak} --- znaki ASCII: HT, LF, CR, LF, SPACE
    \item \textit{cyfra} --- znaki ASCII: 0-9
    \item \textit{mała litera} --- znaki Unicode będące małymi literami 
    \footnote{Wszystkie takie znaki \textbf{X}, że spełniony jest predykat 
        \texttt{char\textunderscore type(X,lower)} w SWI Prologu.}
    \item \textit{wielka litera} --- znaki Unicode będące wielkimi literami
    \footnote{Analogicznie jak wyżej, tylko 
        \texttt{char\textunderscore type(X, upper)}.}
\end{itemize}

\subsection{Struktura leksykalna}

\textit{Komentarz} to ciąg znaków rozpoczynających się znakiem \texttt{\#}, 
zakończony znakiem nowej linii.

\textit{Token} to najdłuższy ciąg znaków niezawierający \textit{komentarzy} ani 
\textit{białych znaków}, taki, że opisuje go poniższa gramatyka:

\begin{grammar}

<literał-całkowitoliczbowy> ::=
    \textit{cyfra} 
    \alt \textit{cyfra} <literał-całkowitoliczbowy>

<literał-znakowy> ::=
    \textbf{'} <znak> \textbf{'}

<literał-napisowy> ::=
    \textbf{\"} <ciąg-znaków> \textbf{\"}

<literał-zmiennopozycyjny> ::=
    <ułamek>
    \alt <ułamek> <wykładnik>

<słowo-kluczowe> ::=
    \textbf{let}
    | \textbf{and}
    | \textbf{in}
    | \textbf{if}
    | \textbf{then}
    \alt \textbf{match}
    | \textbf{else}
    | \textbf{with}
    | \textbf{type}
    | \textbf{of}
    | \textbf{where}
    \alt \textbf{true}
    | \textbf{false}
    | \textbf{import}
    | \textbf{fun}

<operator> ::= <znak-specjalny> | <operator> <znak-specjalny>

<identifikator> ::= 
    \textit{mała litera} <alfanum>, taki, że nie jest <słowo-kluczowe>

<nazwa-typu> ::=
    \textit{wielka litera} <alfanum>

\end{grammar}

gdzie\\

\begin{grammar}

<znak-specjalny> ::=
    \textbf{+}
    | \textbf{\texttt{-}}
    | \textbf{\texttt{*}}
    | \textbf{\texttt{/}}
    | \textbf{\texttt{=}}
    | \textbf{\texttt{:}}
    | \textbf{\texttt{\textgreater}}
    \alt \textbf{\texttt{\textless}}
    | \textbf{\&}
    | \textbf{\texttt{!}}
    | \textbf{\texttt{|}}
    | \textbf{\texttt{\$}}
    | \textbf{\texttt{@}}
    | \textbf{\texttt{\%}}
    | \textbf{\texttt{\^}}
    | \textbf{\texttt{\~}}

    <znak> ::= 
        \textit{dowolny znak Unicode oprócz 
            \textbf{"\\"}, \textbf{"\""}, \textbf{"\'"}}
        \alt \textbf{"\\"n}\,
        | \textbf{"\\"b}\,
        | \textbf{"\\"f}\,
        | \textbf{"\\"a}\,
        | \textbf{"\\"r}\,
        | \textbf{"\\"t}\,
        | \textbf{"\\"v}\,
        | \textbf{"\\\\"}\,
        | \textbf{"\\"'}\,
        | \textbf{"\\\""}\,

    <ciąg-znaków> ::=
        <znak>
        \alt <ciąg-znaków> <znak>
    
    <litera> ::=
        \textit{mała litera}
        \alt \textit{wielka litera}

    <alfanum> ::=
        <litera>
        \alt <alfanum> <litera>
        \alt <alfanum> <cyfra>

    <ułamek> ::=
        <literał-całkowitoliczbowy> \textbf{.} <literał-całkowitoliczbowy>
    
    <wykładnik> ::=
        \textbf{e} \textbf{-} <literał-całkowitoliczbowy>
        \alt \textbf{e} <literał-całkowitoliczbowy>

\end{grammar}

\subsection{Składnia}

\begin{grammar}

<program> ::= 
    <wyrażenie>
    \alt <definicja-typu> <program>
    \alt <import> <program>

<wyrażenie> ::=
    <wyrażenie> \textbf{;} <wyrażenie>
    \alt <definicja> \textbf{in} <wyrażenie>
    \alt \textbf{(} <wyrażenie> \textbf{)}
    \alt <wyrażenie-arytmetyczne>
    \alt <literał-napisowy>
    \alt <literał-znakowy>

    % TODO: TODO: TODO:

\end{grammar}

\section{Semantyka}
\end{document}